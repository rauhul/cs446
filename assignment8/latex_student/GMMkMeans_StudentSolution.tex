\newcommand{\GMMkMeansStudSolA}{
K-means is an un-supervised method that attempts to find hidden\\structure in a set of data.
}

\newcommand{\GMMkMeansStudSolB}{
\begin{align*}
&r_{ik} \in \{0,1\}  &\forall  i \in D , k \in K\\
\sum_{K} &r_{ik} = 1 &\forall  i \in D \\
\end{align*}
}

\newcommand{\GMMkMeansStudSolC}{
\begin{align*}
&r_{ik} \in [0,1]  &\forall  i \in D , k \in K\\
\sum_{K} &r_{ik} = 1 &\forall  i \in D \\
\end{align*}
}

\newcommand{\GMMkMeansStudSolD}{
The best number of clusters for this set of data is 5, as it has the\\minimum euclidean distance between cluster centers and doesn't introduce\\unnecessary clusters.
}

\newcommand{\GMMkMeansStudSolE}{
No, K-means would not be an efficient algorithm to cluster the\\data. K-means attempts to find cluster centers with neat hyper-spheres around\\them; this is due to desire to minimize the intra-cluster sum of squares. Since the\\data doesn't follow this presumed shape (both clusters share the same apparent\\cluster center) the algorithm fails to cluster it well.
}
